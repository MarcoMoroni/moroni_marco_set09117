% #######################################
% ########### FILL THESE IN #############
% #######################################
\def\mytitle{Coursework Report}
\def\mykeywords{algorithms, data structures, Python, checkers, draughts}
\def\myauthor{Marco Moroni}
\def\contact{40213873@live.napier.ac.uk}
\def\mymodule{Algorithms \& Data Structures  (SET09117)}
% #######################################
% #### YOU DON'T NEED TO TOUCH BELOW ####
% #######################################
\documentclass[10pt, a4paper]{article}
\usepackage[a4paper,outer=1.5cm,inner=1.5cm,top=1.75cm,bottom=1.5cm]{geometry}
\twocolumn
\usepackage{graphicx}
\graphicspath{{./images/}}
%colour our links, remove weird boxes
\usepackage[colorlinks,linkcolor={black},citecolor={blue!80!black},urlcolor={blue!80!black}]{hyperref}
%Stop indentation on new paragraphs
\usepackage[parfill]{parskip}
%% Arial-like font
\IfFileExists{uarial.sty}
{
    \usepackage[english]{babel}
    \usepackage[T1]{fontenc}
    \usepackage{uarial}
    \renewcommand{\familydefault}{\sfdefault}
}{
    \GenericError{}{Couldn't find Arial font}{ you may need to install 'nonfree' fonts on your system}{}
    \usepackage{lmodern}
    \renewcommand*\familydefault{\sfdefault}
}
%Napier logo top right
\usepackage{watermark}
%Lorem Ipusm dolor please don't leave any in you final report ;)
\usepackage{lipsum}
\usepackage{xcolor}
\usepackage{listings}
%give us the Capital H that we all know and love
\usepackage{float}
%tone down the line spacing after section titles
\usepackage{titlesec}
%Cool maths printing
\usepackage{amsmath}
%PseudoCode
\usepackage{algorithm2e}

\titlespacing{\subsection}{0pt}{\parskip}{-3pt}
\titlespacing{\subsubsection}{0pt}{\parskip}{-\parskip}
\titlespacing{\paragraph}{0pt}{\parskip}{\parskip}
\newcommand{\figuremacro}[5]{
    \begin{figure}[#1]
        \centering
        \includegraphics[width=#5\columnwidth]{#2}
        \caption[#3]{\textbf{#3}#4}
        \label{fig:#2}
    \end{figure}
}

\lstset{
	escapeinside={/*@}{@*/}, language=C++,
	basicstyle=\fontsize{8.5}{12}\selectfont,
	numbers=left,numbersep=2pt,xleftmargin=2pt,frame=tb,
    columns=fullflexible,showstringspaces=false,tabsize=4,
    keepspaces=true,showtabs=false,showspaces=false,
    backgroundcolor=\color{white}, morekeywords={inline,public,
    class,private,protected,struct},captionpos=t,lineskip=-0.4em,
	aboveskip=10pt, extendedchars=true, breaklines=true,
	prebreak = \raisebox{0ex}[0ex][0ex]{\ensuremath{\hookleftarrow}},
	keywordstyle=\color[rgb]{0,0,1},
	commentstyle=\color[rgb]{0.133,0.545,0.133},
	stringstyle=\color[rgb]{0.627,0.126,0.941}
}

\thiswatermark{\centering \put(336.5,-38.0){\includegraphics[scale=0.8]{logo}} }
\title{\mytitle}
\author{\myauthor\hspace{1em}\\\contact\\Edinburgh Napier University\hspace{0.5em}-\hspace{0.5em}\mymodule}
\date{}
\hypersetup{pdfauthor=\myauthor,pdftitle=\mytitle,pdfkeywords=\mykeywords}
\sloppy
% #######################################
% ########### START FROM HERE ###########
% #######################################
\begin{document}
    \maketitle
    \begin{abstract}
        The goal of this coursework is to implement the classic board game of checkers in an arbitrary computer language demonstrating a correct use of data structures. The language chosen here is Python and the game can be played from the console.
    \end{abstract}
    
    \textbf{Keywords -- }{\mykeywords}
    
    \section{Introduction}
       
    \section{Implementation}
    
    \subsection{Overwiew}
    The board is made by a 2D list, in which every item can be either empty or a piece. \\
    Every piece know its \textit{rank} (man or king) and what player they belong to. \\
    Moves are recorded in a stack called \texttt{moves}, which is used when undoing. There is another similar stack, \texttt{redoMoves}, used to store moves that can be redone. \\
    The game uses a while loop as a game loop. This loop runs until there is a winner. \\
    Finally, the AI works by choosing random moves.
    
    \subsection{Board}
    This project started by creating a simple board made by a 2D array. In Python this is archived by using a list of lists, which means a list of rows where each row is a list os sqaures. \\
    There are 8 rows and 8 columns and each sqaure of the board is initially empty (\texttt{None} in Python).
    
    \subsection{Pieces}
    In checkers a piece can be:
    \begin{itemize}
    	\item either black or white;
    	\item either a man or a king.
    \end{itemize}
	Considering that, the game uses a \texttt{Piece} class that knows:
	\begin{itemize}
		\item its player (of type \texttt{Player});
		\item its \textit{rank}, that is wheter it is a man or a king.
	\end{itemize}
	The ranks are enumerated elements of a class \texttt{PieceRank}: the values are \texttt{PieceRank.MAN} and \texttt{PieceRank.KING}.
	Initially, a piece also knew its position it had at the beginning of the game. This information could have been used in an early version of the \texttt{replay} funtion, where there was a board reset to its inital state. Later on this information became useless, because \texttt{replay} now resets the board by undoing every move.
	
	\subsection{Players}
	A player is normally identified by being either black or white. In this implementation, a player is an instance of a class \texttt{Player}. This class has:
	\begin{itemize}
		\item a dictionary of symbols (\textit{symbols} are the characters printed in the console to represent a piece). The keys are \texttt{PieceRank}s and the values are a character (eg. a white/black dot for men, or a white/black sqaure for kings);
		\item a boolean \texttt{isFacingUp}. This is used in two occasions: when setting up the pieces (see \textit{Prelude}) and when getting all the legal displacements of a piece (see \textit{Getting legal displacements});
		\item a boolean \texttt{cpu};
		\item all the functions used by AI (called when the player is not human) (see \textit{AI}).
	\end{itemize}

	\subsection{Moves}
	...

	
	\section{Formatting}
	Some common formatting you may need uses these commands for \textbf{Bold Text}, \textit{Italics}, and \underline{underlined}. \texttt{Inline code}.
	\subsection{Referencing}
	You should cite References like this: \cite{Keshav}. The references are saved in an external .bib file, and will automatically be added to the bibliography at the end once cited.
	
	\figuremacro{h}{placeholder}{ImageTitle}{ - Some Descriptive Text}{1.0}
	\subsection{LineBreaks}
	Here is a line
    
    Here is a line followed by a double line break.
	This line is only one line break down from the above, Notice that latex can ignore this
    
    We can force a break \\ with the break operator.
    
	\subsection{Maths}
    Embedding Maths is Latex's bread and butter    
    
    {\centering \Large \(
        J = \begin{bmatrix}
            \frac{\delta e}{\delta \theta _0}
            \frac{\delta e}{\delta \theta _1}
            \frac{\delta e}{\delta \theta _2}
        \end{bmatrix}
        = e_{current} - e_{target} 
    \)\par}
	
	\subsection{Code Listing}
    You can load segments of code from a file, or embed them directly.
    
\begin{lstlisting}[caption = Hello World! in c++]
#include <iostream>

int main() {
    std::cout << "Hello World!" << std::endl;
    std::cin.get();
    return 0;
}
\end{lstlisting}

\lstinputlisting[caption = Hello World! in python script]{./sourceCode/hello.py}
    
\subsection{PseudoCode}

\begin{algorithm}[h]
\For{$i = 0$ \KwTo $100$}{
 print\_number = true\;
\If{i is divisible by 3}{
 print "Fizz"\;
 print\_number = false\;
}
\If{i is divisible by 5}{
 print "Buzz"\;
 print\_number = false\;
}
\If{print\_number}{
    print i\;
}
print a newline\;
}
\caption{FizzBuzz}
\end{algorithm}
	
\section{Conclusion}	
\bibliographystyle{ieeetr}
\bibliography{references}
		
\end{document}